\documentclass[a4paper,11pt]{article}
\usepackage[utf8]{inputenc}
\usepackage[italian]{babel}
\usepackage{amsmath}
\usepackage{amsfonts}
\usepackage{amssymb}
\usepackage{physics}
\usepackage{graphicx}
\usepackage{subfig}
\usepackage{hyperref}
\usepackage{parskip}
\usepackage{tabu, wrapfig}
\usepackage[italiano, ruled]{algorithm2e}
\usepackage[left=1in, right=1in]{geometry}

\newcommand{\avg}[1]{\langle {#1} \rangle}
\newcommand{\code}[1]{\texttt{#1}}

\title{Simulazione numerica del modello di Ising 2D}
\author{Rocco Francesco Basta}
\date{}

\begin{document}
	\maketitle
	\section{Introduzione}
		Il modello di Ising 2D consiste in un reticolo di spin, ognuno dei quali può assumere un valore discreto $s_i = \pm 1$ ed interagisce con i suoi primi vicini e, eventualmente, con un campo magnetico esterno.

		L'Hamiltoniana del sistema è data da

		\begin{equation}
			H = -J\sum_{<ij>} s_i s_j - h \sum_i s_i
		\end{equation}

		dove $J > 0$ è la costante di accoppiamento fra primi vicini, mentre $h$ è un campo magnetico esterno.

		Possiamo definire la densità di magnetizzazione $M$, la densità di
		energia $\epsilon$, la suscettività magnetica $\chi$ e il calore
		specifico $C$ del sistema: sia $V$ il volume del reticolo,

		\begin{subequations}
		\begin{equation}
			M \equiv \frac{1}{V} \sum_i s_i
		\end{equation}
		\begin{equation}
			\epsilon \equiv \frac{E}{V}
		\end{equation}
		\begin{equation}
			\chi \equiv \frac{\partial \avg{M}}{\partial h} \propto V ( \avg{M^2} - \avg{M}^2)
		\end{equation}
		\begin{equation}
			C \equiv \frac{\partial \avg{\epsilon}}{\partial T} \propto V (\avg{\epsilon^2} - \avg{\epsilon}^2)
		\end{equation}

		\end{subequations}

		Il modello presenta una transizione di fase del secondo ordine per
		$\beta_c \equiv 1/T_{c} \simeq 0.4407$. Attorno al punto critico, la lunghezza di
		correlazione $\xi$ diverge, e il comportamento del sistema è descritto
		dagli esponenti critici $\alpha, \beta, \gamma, \nu$. Definendo la temperatura ridotta $t \equiv \beta - \beta_c$,

		\begin{subequations}
		\begin{equation}
			\xi \sim |t|^{-\nu} 
		\end{equation}
		\begin{equation}
			\avg{M} \sim |t|^\beta \quad (T < T_c) 
		\end{equation}
		\begin{equation}
			\chi \sim |t|^{-\gamma}
		\end{equation}
		\begin{equation}
			C \sim |t|^{-\alpha}
		\end{equation}
		\label{eqn:t_scaling}
		\end{subequations}

		Gli esponenti critici sono noti esattamente: $\nu = 1$, $\beta = 1/8$,
		$\gamma = 7/4$, $\alpha = 0$.

		Le simulazioni sono effettuate a volume finito. Di conseguenza, $\avg{M}$ non è un buon parametro d'ordine, perché per simulazioni abbastanza lunghe si deve annullare. Al suo posto, si deve studiare $\avg{|M|}$, e anche $\chi$ va misurata calcolando la varianza di $|M|$.
		
		Per non appesantire la scrittura, d'ora in poi indicheremo semplicemente con $M, \epsilon$ le quantità $\avg{|M|}, \avg{\epsilon}$.
		
		Per volumi finiti, $\xi$ non può divergere e diventa confrontabile
		con $L$. Assumendo che nell'intorno della transizione il sistema perda
		memoria del comportamento microscopico (e quindi della spaziatura del
		reticolo),
		si ottengono delle relazioni di scaling per $M$, $\chi$ e $C$:

		\begin{subequations}
		  \begin{equation}
			\chi(\beta, L) = L^{\gamma/\nu} f_\chi (tL^{1/\nu})
		  \end{equation}
		  \begin{equation}
			C(\beta, L) = L^{\alpha/\nu} f_C (tL^{1/\nu})
		  \end{equation}
		  \begin{equation}
			M(\beta, L) = L^{\beta/\nu} f_M(tL^{1/\nu})
		  \end{equation}
		  \label{eqn:fs_scaling}
		\end{subequations}

	\section{Simulazioni numeriche}

	La transizione è stata studiata attraverso un algoritmo Metropolis per
	reticoli di dimensione $N = 20,30,40,50,60$, con $\beta$ compreso tra 0.3 e
	0.505. Per ogni coppia $(N, \beta )$,
	sono state prese $10^{5}$ misure, ognuna ogni spazzata di update, partendo da un reticolo di spin orientati casualmente. 

	La simulazione è stata effettuata in assenza di campo magnetico esterno
	($h = 0$), e fissando $J = 1$.

	Gli errori su $\epsilon$ e $M$ sono stati
	stimati attraverso un processo di blocking, mentre gli errori su $\chi$ e $C$ sono stati stimati attraverso un algoritmo Bootstrap.

	Il generatore di numeri casuali utilizzato è \code{RAN2}, tratto dalle \emph{Numerical Recipes for C}.
	
	\subsection{Misure effettuate}
	
	Sono riportati in figura \ref{fig:em_plot} i grafici di $M, \epsilon$ in funzione di $\beta$ ottenuti nelle simulazioni. In figura \ref{fig:chiC_plot} invece è riportato l'andamento di $\chi$ e di $C$.
	
	\begin{figure}
        \subfloat[$M$]{\includegraphics[width=0.5\textwidth]{figure/m_plot.pdf}}
        \subfloat[$\epsilon$]{\includegraphics[width=0.5\textwidth]{figure/e_plot.pdf}}
        \caption{Magnetizzazione e densità di energia in funzione di $\beta$.}
        \label{fig:em_plot}
	\end{figure}
	
	\begin{figure}
        \subfloat[$\chi$]{\includegraphics[width=0.5\textwidth]{figure/chi_plot.pdf}}
        \subfloat[$C$]{\includegraphics[width=0.5\textwidth]{figure/C_plot.pdf}}
        \caption{Suscettività magnetica e capacità termica in funzione di $\beta$.}
        \label{fig:chiC_plot}
	\end{figure}
	
	Possiamo utilizzare i valori teorici degli indici critici per verificare le relazioni di scaling (eq. \ref{eqn:fs_scaling}). Il risultato è riportato in figura \ref{fig:mchiC_scaling_plot}. Per ottenere il collasso per la capacità termica, è stato necessario sottrarre un termine di fondo che non diverge attorno al punto critico. Per semplicità, questo è stato fatto sottraendo a $C$ il suo massimo, per ogni $N$, prima di applicare la relazione di scaling. Questo metodo non è pienamente soddisfacente, tuttavia il risultato è abbastanza buono nei pressi di $\beta_c$. 
	
	\begin{figure}
        \subfloat[$M$]{\includegraphics[width=0.5\textwidth]{figure/m_scaling.pdf}}
        \subfloat[$\chi$]{\includegraphics[width=0.5\textwidth]{figure/chi_scaling.pdf}} \\
        \subfloat[$C$]{\includegraphics[width=0.5\textwidth]{figure/C_scaling.pdf}}
        \caption{Verifica dello scaling di size finito per $M$, $\chi$, $C$ utilizzando i valori teorici degli indici critici.}
        \label{fig:mchiC_scaling_plot}
	\end{figure}

	
	\subsection{Temperatura critica}
	
    A N finito, il massimo di $\chi$ non corrisponde a $\beta_c$, ma a un valore inferiore, detto $\beta_{pc}$ ($\beta$ pseudocritico). Lo stesso avviene per $C$, ad una diversa temperatura $\beta'_{pc}$.
    
    Dalle equazioni (\ref{eqn:t_scaling}) e (\ref{eqn:fs_scaling}) è facile dimostrare che $\beta_{pc}$ soddisfa una relazione analoga
    
    \begin{equation}
        \beta_{pc} = \beta_c + x N^{-1/\nu}
        \label{eqn:bpc_scaling}
    \end{equation}
    
    da cui possiamo ricavare $\beta_c$ e $\nu$.
    
    Alle misure di $\beta_{pc}$ ottenute considerando il massimo della suscettività, è stato associato un errore $\Delta \beta_{pc}$ pari a metà della distanza tra due misure consecutive ($\Delta \beta_{pc} = 0.0025$).
    
    Non è stato possibile effettuare un fit numerico alla funzione (\ref{eqn:bpc_scaling}) lasciando liberi tutti e tre i parametri $\beta_c$, $x$ e $\nu$, per problemi di convergenza del fit.
    
    Tuttavia, un fit con $\nu = 1$ fissato fornisce $\beta_c = 0.4395(29)$, che è compatibile con il valore teorico, con un $\chi^2 / \text{ndof} \simeq 1.24$. Il grafico è riportato nella figura \ref{fig:bc_fit}.
    
    \begin{figure}
        \centering
        \includegraphics[width=0.7\textwidth]{figure/fit_bpc.pdf}
        \caption{Misura di $\beta_c$ dal fit di $\beta_{pc}(N)$.}
        \label{eqn:bpc_scaling}
    \end{figure}


	
	\subsection{Scaling rispetto alla temperatura}
	
	Come prima cosa, si è tentato di estrarre gli indici critici da un fit analitico di $\chi, C, M$ in funzione di $t$, vicino alla transizione.

	I fit sono stati effettuati con le misure ottenute dal reticolo più grande ($N = 60$), e sono stati eseguiti in entrambe le regioni scalanti $t > 0$ e $t < 0$ (tranne $M$, perché la relazione di scaling vale solo per $T < T_c$).
	
	Il fatto che $\alpha = 0$ implica $C \sim \log(t)$ nella regione scalante.
	
	I fit per $\chi$ e $C$ sono riportati nelle figure \ref{fig:chi_fit} e \ref{fig:C_fit}. 
	
	\begin{wraptable}{r}{5.5cm}
        \begin{tabular}{c c c} \hline 
                & $\gamma$  & $\chi^2 / \text{ndof}$ \\ \hline
                $\beta > \beta_c$ & 1.783(53) & 1.77 \\
                $\beta < \beta_c$ & 1.743(45) & 0.76  \\ \hline
        \end{tabular}
	\end{wraptable}
	
	\begin{wraptable}{r}{5.5cm}
        \begin{tabular}{c c c} \hline 
                & $C$  & $\chi^2 / \text{ndof}$ \\ \hline
                $\beta > \beta_c$ & 1.783(53) & 1.77 \\
                $\beta < \beta_c$ & 1.743(45) & 0.76  \\ \hline
        \end{tabular}
	\end{wraptable}

	
	\begin{figure}
        \centering
        \subfloat[$\beta > \beta_c$]{\includegraphics[width=0.5\textwidth]{figure/chi_fit.pdf}}
        \subfloat[$\beta < \beta_c$]{\includegraphics[width=0.5\textwidth]{figure/chi_fit_rev.pdf}}
        \caption{Fit della suscettività magnetica $\chi$ nella regione scalante.}
        \label{fig:chi_fit}
	\end{figure}

	\begin{figure}
        \centering
        \subfloat[$\beta > \beta_c$]{\includegraphics[width=0.5\textwidth]{figure/C_fit.pdf}}
        \subfloat[$\beta < \beta_c$]{\includegraphics[width=0.5\textwidth]{figure/C_fit_rev.pdf}}
        \caption{Fit della capacità termica $C$ nella regione scalante.}
        \label{fig:C_fit}
	\end{figure}

	\subsection{Analisi di size finito per $\beta = \beta_c$}

\end{document}
